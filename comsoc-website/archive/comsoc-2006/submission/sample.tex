% Sample paper for COMSOC-2006
\documentclass{comsoc}

%%%%%%%%%%%%%%%%%%%%%%%%%%%%%%%%%%%%%%%%%%%%%%%%%%%%%%%%%%%%%%%%%%%%%%%%%

% Specify your title and author names here:
\title{\bf Paper Submissions to COMSOC-2006}
\author{Ulle Endriss and J\'er\^ome Lang}

%%%%%%%%%%%%%%%%%%%%%%%%%%%%%%%%%%%%%%%%%%%%%%%%%%%%%%%%%%%%%%%%%%%%%%%%%

% Page numbers are useful for submission, but need to be removed during 
% preparation of the camear-ready version. Remove the following line to 
% get rid of page numbers:
\pagestyle{plain}

%%%%%%%%%%%%%%%%%%%%%%%%%%%%%%%%%%%%%%%%%%%%%%%%%%%%%%%%%%%%%%%%%%%%%%%%%

\begin{document}

%%%%%%%%%%%%%%%%%%%%%%%%%%%%%%%%%%%%%%%%%%%%%%%%%%%%%%%%%%%%%%%%%%%%%%%%%

% Include a short abstract here (100-300 words):
\begin{abstract}
This short paper explains the formatting instructions for submissions
to the 1st International Workshop on Computational Social Choice.
\end{abstract}

%%%%%%%%%%%%%%%%%%%%%%%%%%%%%%%%%%%%%%%%%%%%%%%%%%%%%%%%%%%%%%%%%%%%%%%%%

\section{Formatting Instructions}

Authors are invited to submit full papers not exceeding 14 pages, 
including references (roughly 5000 words). Each paper should include 
a title, the names and contact details of all authors, and an abstract 
of 100--300 words.

Please format your paper according to the following guidelines.
The most important requirements are (1)~that the submission should be 
formatted for A4 paper (that is, the page size as shown under 
\emph{Document Properties} in the Acrobat Reader, for instance, should 
be $8.27\times 11.69\,\mbox{in}$); and (2)~that the text should fit into 
an area of $4.75\times 7.5\,\mbox{in}$ ($12.07\times 19.05\,\mbox{cm}$). 
This excludes page numbers (which we suggest you include for the 
submission, but which must be removed for the camera-ready version in 
case of acceptance). Please use a 10pt typeface, with suitable deviations
for section headings, footnotes, etc. In general, please aim at having 
your paper look as close to this sample as possible. The easiest way of 
achieving this is to use the Latex document preparation system with the 
style file \texttt{comsoc.cls} provided at the workshop  website (take 
the file \texttt{sample.tex} as a starting point).

Papers not conforming to these guidelines will be accepted for review
(provided they are not excessively long), but in case of acceptance we
must insist that the guidelines be followed during preparation of the 
camera-ready version.  

\section{What is Computational Social Choice?}

Computational social choice is an interdisciplinary field of study at the
interface of social choice theory and computer science, promoting an 
exchange of ideas in both directions. On the one hand, it is concerned 
with the application of techniques developed in computer science, such as
complexity analysis or algorithm design, to the study of social choice 
mechanisms, such as voting procedures or fair division algorithms. On the
other hand, computational social choice is concerned with importing 
concepts from social choice theory into computing. For instance, social 
welfare orderings originally developed to analyse the quality of resource
allocations in human society are equally well applicable to problems in 
multiagent systems or network design.

Social choice theory is concerned with the design and analysis of methods
for collective decision making. Much classical work in the field has 
concentrated on establishing abstract results regarding the existence 
(or otherwise) of procedures meeting certain requirements, but such work 
has not usually taken computational issues into account. For instance, 
while it may not be possible to design a voting protocol that makes it 
impossible for a voter to cheat in one way or another, it may well be 
the case that cheating successfully turns out to be a computationally 
intractable problem, which may therefore be deemed an acceptable risk.

Examples for topics studied in computational social choice include the 
complexity-theoretic analysis of voting protocols (both with a view to 
developing computationally feasible mechanisms, and to exploit 
computational intractability as a means against strategic manipulation); 
the formal specification and verification of social procedures such as 
fair division algorithms (social software); and the application of 
techniques developed in artificial intelligence and logic to the compact 
representation of preferences in combinatorial domains (such as 
negotiation over indivisible resources or voting for assemblies).

%%%%%%%%%%%%%%%%%%%%%%%%%%%%%%%%%%%%%%%%%%%%%%%%%%%%%%%%%%%%%%%%%%%%%%%%%

% References (automatically generated using Bibtex)

\begin{thebibliography}{1}

\bibitem{BramsTaylor1996}
S.~J. Brams and A.~D. Taylor.
\newblock {\em Fair Division: {F}rom Cake-cutting to Dispute Resolution}.
\newblock Cambridge University Press, 1996.

\bibitem{NisanEC2000}
N.~Nisan.
\newblock Bidding and allocation in combinatorial auctions.
\newblock In {\em Proc.\ 2nd ACM Conference on Electronic Commerce (EC-2000)}.
  ACM Press, 2000.

\bibitem{PiniEtAlTARK2005}
M.~S. Pini, F.~Rossi, K.~B. Venable, and T.~Walsh.
\newblock Aggregating partially ordered preferences: {I}mpossibility and
  possibility results.
\newblock In {\em Proc.\ 10th Conference on Theoretical Aspects of Rationality
  and Knowledge (TARK-2005)}. National University of Singapore, 2005.

\bibitem{RusinowskaEtAlSCW2005}
A.~Rusinowska, H.~de~Swart, and J.-W. van~der Rijt.
\newblock A new model of coalition formation.
\newblock {\em Social Choice and Welfare}, 24(1):129--154, 2005.

\end{thebibliography}

%%%%%%%%%%%%%%%%%%%%%%%%%%%%%%%%%%%%%%%%%%%%%%%%%%%%%%%%%%%%%%%%%%%%%%%%%

% At the very end of the paper, please include your contact details:

\begin{contact}
Ulle Endriss \\
ILLC, University of Amsterdam \\
1018 TV Amsterdam, The Netherlands \\
\email{ulle@illc.uva.nl}
\end{contact}

\begin{contact}
J\'er\^ome Lang \\
IRIT, Universit\'e Paul Sabatier \\
31062 Toulouse Cedex, France \\
\email{lang@irit.fr}
\end{contact}

%%%%%%%%%%%%%%%%%%%%%%%%%%%%%%%%%%%%%%%%%%%%%%%%%%%%%%%%%%%%%%%%%%%%%%%%%

\end{document}

%%%%%%%%%%%%%%%%%%%%%%%%%%%%%%%%%%%%%%%%%%%%%%%%%%%%%%%%%%%%%%%%%%%%%%%%%


